\documentclass{article}

\usepackage[spanish]{babel}

% Required for inserting images
\usepackage{graphicx} 

\usepackage{listings}

\usepackage[utf8]{inputenc}
\usepackage{glossaries}
\usepackage{csquotes}
\usepackage{parskip}

%Imports biblatex package
\usepackage{biblatex}

\usepackage{hyperref}
\hypersetup{
    colorlinks,
    citecolor=black,
    filecolor=black,
    linkcolor=black,
    urlcolor=black
}

% import the bibliography file
\addbibresource{misticaMusicaBib.bib}

\setlength{\parindent}{0pt}

\title{Mística y música algorítmica}

\author{Aarón Montoya-Moraga}

\date{Julio 2025}

\begin{document}

\maketitle

\renewcommand*\contentsname{Tabla de contenidos}

\tableofcontents

\clearpage

\section{Introducción}

Este documento es el resultado de la investigación que realicé durante el seminario de investigación \textit{Mística y música}, dictado por el profesor y doctor Felipe Cussen en IDEA USACH, en el marco del primer semestre del Doctorado de Artes y Humanidades, entre marzo y julio 2025.

Enfrenté este curso desde mi gran ignorancia, que iba desde la diferencia entre mística y religión, la historia sufí y mis superficiales nociones sobre el movimiento \textit{new age}.

En este texto primero describo mis definiciones sobre mística, música y algoritmos, basándome en la bibliografía del curso. 

A continuación documento el trabajo y la experiencia detrás de mi participación en el concierto "Acordes Místicos" en la Iglesia de la Veracruz, junto a las colegas del curso.

En este concierto apliqué mi aproximación sensible al uso de la guitarra eléctrica como una pieza fundamental de mi práctica artística, sonora y musical.

Este es el tercero de los tres textos que escribí como cierre de los tres cursos del primer semestre del doctorado, porque me iba ser el más difícil de aterrizar.

Durante la escritura de este texto escuché las obras Ecstatic Computation de Caterina Barbieri (2019), Music for 18 Musicians de Steve Reich (1978) y el disco Scriabin Sonatas Nos 1-10, interpretadas por el pianista Anatol Ugorski y publicadas por el sello Deutsche Gramophon(2023).

\clearpage

\section{Definición de mística}


\clearpage

\section{Definición de música}

Llevo una década de trabajo sonoro desmarcándome de la música, prefiriendo abordar la etiqueta de arte sonoro o producción.

\clearpage

\section{Definición de algorítmica}

La algorítmica la definiremos para efectos de este texto como una técnica consistente en pasos, repeticiones, iteraciones.

Un algoritmo es una secuencia de instrucciones, con un inicio, quizás sin un final. Estas instrucciones pueden repetirse, cambiar su flujo según variables, incluso incluir aleatoreidad.

La primera vez que me enfrenté al concepto de algoritmo fue en ciencias de la computación, pero esta práctica existe en artes de mi interés, como artes visuales y artes sonoros.

A nivel de artes visuales, quiero destacar el caso de Sol Lewitt.

A nivel de artes sonoras, quiero destacar a Steve Reich, quien 

Serialismo

\clearpage


\section{Conclusiones y pasos a seguir}

Espero tras este curso continuar con tres aspectos desarrollados durante el curso:

\begin{enumerate}
    \item El uso de la guitarra eléctrica en escenario
    \item La exploración colectiva en ensambles grandes, a partir de lenguajes musicales contemporáneos
    \item El uso de referentes precisos sobre los cuales desarrollar mi obra.
\end{enumerate}



\clearpage

\printbibliography[title={Bibliografía}, heading=bibintoc]

\end{document}