\documentclass{article}

\usepackage[spanish]{babel}

% Required for inserting images
\usepackage{graphicx} 

\usepackage[utf8]{inputenc}
\usepackage{glossaries}
\usepackage{csquotes}
\usepackage{parskip}

%Imports biblatex package
\usepackage{biblatex}

\usepackage{hyperref}
\hypersetup{
    colorlinks,
    citecolor=black,
    filecolor=black,
    linkcolor=black,
    urlcolor=black
}

% import the bibliography file
\addbibresource{investigacion.bib}

\setlength{\parindent}{0pt}

\title{Una propuesta para la administración computacional de traiciones}
\author{Aarón Montoya-Moraga}

\date{Julio 2025}

\begin{document}

\maketitle

\renewcommand*\contentsname{Tabla de contenidos}

\tableofcontents

\clearpage

\section{Introducción}

Este documento es el resultado de la investigación que realicé durante el seminario de investigación \textit{Traidores en las dictaduras del cono sur}, dictado por el profesor y doctor José Santos Herceg en IDEA USACH, en el marco del primer semestre del Doctorado de Artes y Humanidades, entre marzo y julio 2025.

La traición es el tema principal de nuestra literatura y nuestra historia \cite[p. 138]{ordinaryVicesShklar}, y en este texto narro la investigación que desarrollé al respecto.

La primera es una aproximación algorítmica desde la ingeniería de software, la segunda es una propuesta de hardware, y la tercera es una app y mínimamente funcional, cuyo mayor avance fue el crear una propuesta de base de datos.

Es mi sueño que la traición y la computación nos permitan lidiar con la traición de formas mas amenas y eficientes.

\clearpage

\section{Definición de traición}

Para definir traición, en este seminario de investigación estudiamos una nutrida bibliografía de autores, entre las que destaco los siguientes enfoques:



\clearpage

\section{Definición de administración}

\clearpage

\section{Definición de traición computacional}

La traición electrónica implica que la electrónica tenga agencia de la electrónica.

En la electrónica podemos cometer errores, podemos también tener promesas incumplidas, como un cable que nos promete hacer una conexión eléctrica, pero que está roto por dentro y no lo sabemos. Pero no existe la voluntad del dispositivo o material de traicionarnos, no toma la decisión de romper nuestra confianza, por lo que no califica como traición.

La traición computacional enfrenta el mismo paradigma, el computador trata de cumplir nuestras promesas, pero no siempre lo logra, ya que comete errores, pierde alimentación, o porque creímos haber depositado nuestra confianza en instrucciones claras, pero no lo eran.

definición de traición con bibliografía y su manejo en electrónica y computación

\clearpage

\section{Administración de traición computacional}

ontología orientada a objetos

los computadores son herramientas que pueden almacenar y procesar datos.

la traición la podemos definir como una estructura de datos, con un conjunto de operaciones para hacer transiciones de estado.

## estado del arte de computadores personales de uso específico

A principios del siglo XX la educación computacional para crear herramientas para artistas y diseñadores era muy limitada, centrándose principalmente en el uso de herramientas de software construidas por la industria.

Gracias al trabajo de Casey Reas y Ben Fry, con su herramienta Processing, creada al alero del laboratorio aesthetics and computation de MIT Media Lab, se logró popularizar el uso de software, para que artistas pudieran crear sus propias herramientas computacionales, bajo el paradigma de este framework de programación en Java.

De hecho Casey Reas fue uno de los co-advisors de la tesis de magíster de Hernando Barragán, quien creó el lenguaje de programación Wiring, que luego tras una traición de otro profesor conformó el proyecto Arduino, que popularizó el uso de microcontroladores para crear máquinas electrónicas.

Revisar el trabajo de Hernando Barragán, <https://arduinohistory.github.io/>

Hoy en día el flujo de artistas creando máquinas para sus obras tienen varios caminos posibles:

1. Correr código desde un computador de escritorio o laptop, con sistema operativo GNU/Linux, macOS o Windows.
1. Correr código en un computador de uso personal dedicado a instalaciones u obra, como una Raspberry Pi, o un Raspberry Pi Computer Module.
  1.1. Ejemplos: computador Norns de monome.
1. Programar un microcontrolador, Arduino o similar, donde el código corra directamente, sin necesidad de un computador más potente.

\section{Traición y software}

La revisión bibliográfico de softwares que contenga la palabra traición no ha dado resultados.

softwares para administrar confianza.

softwares de uso sensible para periodistas: Proyecto Guardian <https://guardianproject.info/>.

Protocolo de encriptamiento <https://signal.org/>

La promesa del software SecureDrop es Share and accept documents securely. <https://securedrop.org/>

Traición por parte de SourceForge: [<https://www.theregister.com/2013/12/20/sourceforge_betrayal/>.](https://arstechnica.com/information-technology/2015/06/sourceforge-locked-in-projects-of-fleeing-users-cashed-in-on-malvertising/)

\clearpage

\section{Traición y hardware}


A nivel de traición con hardware, queremos destacar el trabajo del comediante Nathan Fielder, quien en su programa de televisión "Nathan for You" comisión la creación de un dispositivo llamado The Claw of Shame <https://www.imdb.com/title/tt2780878/>.

Este dispositivo es una garra mecánica capaz de bajarle sus pantalones en público. Su contexto de uso es probar al usuario si es capaz de hacer una destreza y desactivar el dispositivo contra reloj, ya que si el dispositivo se activa, el usuario queda expuesto a la vergüenza pública y las repercusiones legales de estar sin pantaleones en público.

\clearpage

\section{Diseño y desarrollo de software administrador de traiciones}

Antes de crear un computador unipropósito, con toda su complejidad de programar, se decidió por crear un software que permita administrar con una interfaz gráfica y textual.

Se usó la biblioteca pygame para crear la interfaz gráfica.

Se usó una tipografía monoespaciada.

Debido a su flexibilidad de uso y legibilidad, se decidió usar el software Python para el primer prototipo de administrar traiciones.

Luego se propone el porte a C++ para que pueda correr un microcontrolador Raspberry Pi Pico 2.

durante esta investigación diseñé clases para definir traicion y otras clases asociadas, con los siguientes atributos y métodos:

atributos:

* nombre
* fecha de creación
* fecha de modificación
* estado (confianzaDepositada, confianzaRetirada, traicion)
* receptor
* emisor
* descripción

metodos:

* depositarConfianza()
* retirarConfianza()
* traicionar()

como inspiración usamos los dispositivos computacionales de monome, como norns, y los de Critter and Guitari, como Organelle y Eyesy.

Se propone este administrador de traiciones como un dispositivo que permite a los usuarios registrar y administrar sus traiciones, tanto las recibidas como las emitidas.

Para la visualización se proponen varios métodos:

el primer prototipo ocurre en una interfaz hecha en Python, con la biblioteca pygame.

el siguiente prototipo es un aparato standalone, con una pantalla LCD y botones y teclado para navegar las traiciones.

Para hardware proponemos un microcontrolador Raspberry Pi Pico 2, basado en el microcontrolador RP2050, de bajo costo y bajo consumo.

Como salida, proponemos una pantalla LCD de 128x64 pixeles, para mostrar texto.

\clearpage

\section{Conclusiones y pasos a seguir}


La primera aproximación de este software fue la investigación de

Luego

Esta estructura de datos y sus operaciones puede ahora ser testeada y mejorada según feedback de los usuarios.

se propone construir una primera versión de los dispositivos físicos.

Durante mi investigación doctoral me enfoco en diseño y construcción de instrumentos musicales electrónicos y computacionales.

Para defender este lugar transdisciplinar, necesito crear comunidad, por eso con mucha alegría quiero compartir que en 2023 y 2024 en mi rol de profesore asistente participé de la creación de la nueva malla curricular de la Escuela de Diseño de la Universidad Diego Portales. Quiero destacar que creamos la nueva mención de diseño en interacción digital, para abordar los intereses disciplinares locales y mundiales en el diseño de interfaces electrónicas, de experiencias de usuario, de robótica y software.

Esta malla se implementó este 2025, y el próximo año cuando les estudiantes estén en su tercer semestre tendrán un nuevo curso: Pensamiento computacional, donde aprenderán los fundamentos y rudimentos de la programación, a la par de herramientas disciplinares tan importantes como lo son el dibujo, la tipografía y la geometría.

En este contexto también comparto que este año 2025 empecé a colaborar sistemáticamente en mi labor académica y profesional con el artista Matías Serrano \cite{misaa}, artista y académico chileno.

El primer semestre del año 2025 dictamos en conjunto un curso semestral en la misma Escuela de Diseño UDP: DIS8644 Taller de diseño de máquinas electrónicas. Las estudiantes aprendieron a sobre componentes básicos eléctricos como resistores y capacitores, y cómo valerse de chips de \textit{through-hole technology} \cite{tht}

En el segundo semestre de este mismo año continuaremos nuestra labor conjunta con el curso hermano DIS8645 Taller de diseño de máquinas computacionales. En este curso nos enfocaremos en la disciplina de construir placas con \textit{surface-mount technology} \cite{smt}, y con microcontroladores programados en el lenguaje C\texttt{++}.

Este julio y agosto 2025, creo que debido a esta nueva exposición en redes sociales de este trabajo, con Matías tomamos un cliente para el que estamos desarrollando un dispositivo electromecánico para percusión robótica. Gracias a las nuevas generaciones de diseñadoras y artistas que hemos ayudado a formar, gran parte de este proyecto particular está siendo producida y remunerada a generaciones más jóvenes de estudiantes, que esperamos se independicen y profesionalicen en artes electrónicas y computacionales.

Creo que es fundamental la creación de nuevos espacios curriculares donde les estudiantes se expongan a las artes electrónicas y computacionales, y así existan audiencias locales aún más robustas que sepan apreciar estas prácticas.

Esta práctica transdisciplinar que desarrollo hoy desde Chile en el año 2025 ha sido quizás anecdótica y definitivamente privilegiada, y espero que con mi labor académica y artística pueda ser la norma y con menos barreras, para quien así lo desee.

\clearpage

\printbibliography[title={Bibliografía}, heading=bibintoc]

\end{document}