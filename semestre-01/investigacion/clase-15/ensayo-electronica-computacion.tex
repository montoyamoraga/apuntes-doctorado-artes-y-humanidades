\documentclass{article}

\usepackage[spanish]{babel}

% Required for inserting images
\usepackage{graphicx} 

\usepackage[utf8]{inputenc}
\usepackage{glossaries}
\usepackage{csquotes}
\usepackage{parskip}

%Imports biblatex package
\usepackage{biblatex}

\usepackage{hyperref}
\hypersetup{
    colorlinks,
    citecolor=black,
    filecolor=black,
    linkcolor=black,
    urlcolor=black
}

% import the bibliography file
\addbibresource{investigacion.bib}

\setlength{\parindent}{0pt}

\title{Balbuceos sobre arte electrónico y computacional como un lugar transdisciplinar}
\author{Aarón Montoya-Moraga}

\date{Julio 2025}

\begin{document}

\maketitle

\renewcommand*\contentsname{Tabla de contenidos}

\tableofcontents

\section{Introducción}

Este ensayo es la primera versión de un manifiesto / matriz de investigación hecha en el marco del curso de investigación inter y transdisciplinar, dictador por las profesoras y doctoras Cynthia Shuffer y Carolina Pizarro, en IDEA USACH, en el marco del primer semestre del Doctorado de Artes y Humanidades.



\section{Electricidad y magneitsmo}

La electricidad es una teología, en el sentido de que nunca nadie ha visto un electrón, no hay imágenes de electrones, al menos hasta hoy.

Las ecuaciones de Maxwell resumen las leyes de Faraday entre otres, en cuatro ecuaciones que muestran cómo la electricidad produce magnetismo, y el magnetismo produce electricidad.

\section{Electrónica}

Cuando usamos la electricidad para procesar no solamente energía, sino que tambieén información, es cuando hablamos de electrónica.

\section{Arte electrónico}

Definamos el arte electrónico como el lugar geométrico creado por el producto cruz entre los dos vectores ortogonales arte y electrónico.

\section{Instrumentos para arte electrónico}

\section{Academia y disciplina de arte electrónico}

¿Cómo entiendo la tensión entre disciplinas académicas y saberes arraigados?

\section{Investigación situada}

La práctica que ejerzo está fuertemente situada desde Latinoamérica en el siglo XXI.

Los artefactos que estudio existen desde el siglo XX.

El ecosistema que habito tiene que ver con electricidad doméstica, con baterías recargables, con desarrollo de chips de silicio, con la base de las leyes de Maxwell y de la computación descrita por Ada Lovelace y Alan Turing.

¿Cómo comprendo la investigación a través de la práctica?

La investigación es una caminata a través de caminos que se documentan como mejor podemos, y quedan como bitácora de lo recorrido.



\section{Colaboración y fuente abierta}

¿Cómo comprendo el movimiento disciplinar que tiene que ver con la colaboración y/o la difuminación de las fronteras disciplinares?

Si hasta Inti-Illimani se separa en facciones, qué queda para el resto.

\section{Conclusiones y pasos a seguir}

Ejemplos de citas de libros.

Citar libro \cite{korgBerlin}


\printbibliography[title={Bibliografía}, heading=bibintoc]

\end{document}