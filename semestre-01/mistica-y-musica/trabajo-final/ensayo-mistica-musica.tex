\documentclass{article}

\usepackage[spanish]{babel}

% Required for inserting images
\usepackage{graphicx} 

\usepackage{listings}

\usepackage[utf8]{inputenc}
\usepackage{glossaries}
\usepackage{csquotes}
\usepackage{parskip}

%Imports biblatex package
\usepackage{biblatex}

\usepackage{hyperref}
\hypersetup{
    colorlinks,
    citecolor=black,
    filecolor=black,
    linkcolor=black,
    urlcolor=black
}

% import the bibliography file
\addbibresource{misticaMusicaBib.bib}

\setlength{\parindent}{0pt}

\title{Mística y música algorítmica}

\author{Aarón Montoya-Moraga}

\date{Julio 2025}

\begin{document}

\maketitle

\renewcommand*\contentsname{Tabla de contenidos}

\tableofcontents

\clearpage

\section{Introducción}

Este documento es el resultado de la investigación que realicé durante el seminario de investigación \textit{Mística y música}, dictado por el profesor y doctor Felipe Cussen en IDEA USACH, en el marco del primer semestre del Doctorado de Artes y Humanidades, entre marzo y julio 2025.

Enfrenté este curso desde mi gran ignorancia, que iba desde la diferencia entre mística y religión, la historia sufí y mis superficiales nociones sobre el movimiento \textit{new age}.

En este texto primero describo mis definiciones sobre mística, música y algoritmos, basándome en la bibliografía del curso. 

A continuación documento el trabajo y la experiencia detrás de mi participación en el concierto "Acordes Místicos" en la Iglesia de la Veracruz, junto a las colegas del curso.

En este concierto apliqué mi aproximación sensible al uso de la guitarra eléctrica como una pieza fundamental de mi práctica artística, sonora y musical.

Este es el tercero de los tres textos que escribí como cierre de los tres cursos del primer semestre del doctorado, porque me iba ser el más difícil de aterrizar.

Durante la escritura de este texto escuché las obras Ecstatic Computation de Caterina Barbieri (2019), Music for 18 Musicians de Steve Reich (1978) y el disco Scriabin Sonatas Nos 1-10, interpretadas por el pianista Anatol Ugorski y publicadas por el sello Deutsche Gramophon(2023).

\clearpage

\section{Definición de mística}


\clearpage

\section{Definición de música}

Llevo una década de trabajo en sonido desmarcándome de la música, prefiriendo abordar la etiqueta de arte sonoro.

Como arte sonoro entiendo el arte que habita en el sonido, esto es, oscilaciones de presión que al ser percibidas por el oído y el cuerpo en general, y luego procesadas por nuestro sistema auditivo.

La música viene a ser un subconjunto de las infinitas posibilidades del arte sonoro, aplicándole una capa más definida y restrictiva de cultura.

Estas definiciones pueden hacerse metáfora, siendo el arte sonoro una entrada a un filtro llamado cultura, y a cuya salida le llamamos música.

Durante este curso investigué nuevamente sobre música, esto es, cultura. Las primeras lecturas y ejemplos vistos en clases sobre música sufí fueron reveladores, ya que me permitieron entender las influencias detrás artistas como La Monte Young (no confundir con La Meme Young).

Estudiar las influencias cercanas y lejanas de artistas que admiro ha sido central en la difuminación de mi educación sonora. Tras disfrutar a The Strokes y Sonic Young, al investigar sus referentes conocí a The Velvet Underground y Glenn Branca, y a su vez la música \textit{drone}, que fuertemente influenció mis presentaciones en vivo con guitarra eléctrica y electrónica.

Cada artista y su obra componen un organismo complejo y en constante cambio, se resisten a definiciones, y nos desafían en este movimiento a tratar de encasillarlos en estilos particulares o movimientos.

Es extremedamente difícil que ponerse de acuerdo entre personas, aún más lo es explicar el sonido usando texto, y quizás imposible por definición, transmitir lo vivido durante una experiencia mística.

\clearpage

\section{Definición de algorítmica}

La algorítmica la definiremos para efectos de este texto como un adjetivo con límites difusos que nos remite a pasos, secuencias, repeticiones, iteraciones y variables.

Un algoritmo entonces es una secuencia de instrucciones, con un inicio, quizás sin un final. Estas instrucciones pueden repetirse, cambiar su flujo según variables, incluso incluir aleatoreidad u otros procesos de manipulación.

La primera vez que me enfrenté al concepto de algoritmo fue en ciencias de la computación, pero esta práctica existe en artes de mi interés, como la performance, las artes visuales y las artes sonoras.

Como artista visual y performer, Yoko Ono en su libro Grapefruit escribe algoritmos a seguir, que no necesitan de computadores ni tampoco ser realizados para convertirse en obra, como este:

\begin{quote}
  Find a stone that is your size or weight.
  
  Crack it until it becomes fine powder.
  
  Dispose of it in the river. (a)

  Send small amounts to your friends. (b)

  Do not tell anybody what you did.

  Do not explain about the powder to the friends to whom you send.\cite{grapefruitYokoOno}
\end{quote}

Otro artista que admiro por su capacidad algorítmica, también sin computadores, es Sol Lewitt, con su serie de obras Wall Drawings:

\begin{quote}
    A wall divided vertically into six equal parts, with two of the four kinds of line directions superimposed in each part.
\end{quote}

A nivel de artes visuales, quiero destacar el caso de Sol Lewitt, quien define. 



A nivel de artes sonoras, quiero destacar a Steve Reich, quien 

Serialismo

\clearpage

\section{Concierto de acordes místicos}

Ensayo

Mezcla

Exploraciones abandonadas en el camino

Debido a la decisión de nombrar a nuestro concierto como \textit{Acordes místicos}, decidí enfocar mi investigación y la pieza musical preparada en ese concepto.

Partí pensando en distintas estrategias armónicas para lograr construir en vivo un acorde místico a partir de una nota \textit{drone}, lo que me llevó a explorar distintas estrategias para lograr un drone.

Quería poder partir desde alguna nota en particular, como Do o La, para poder darle esta nota al resto de los integrantes del ensamble para poder sumarse si así lo deseaban durante mi pieza.

Partí probando con mi voz como instrumento, pero para esta ocasión lo deseché por mi dificultad en afinarme a una nota en particular. El siguiente experimento consistió en usar un \textit{Ebow} para lograr la nota drone, pero este dispositivo me ocupaba una mano, y prefería tener las manos libres para manipular otros parámetros.



Whammy

monome norns

Guitarra eléctrica



\section{Conclusiones y pasos a seguir}

Espero tras este curso continuar con tres aspectos desarrollados durante el curso:

\begin{enumerate}
    \item El uso de la guitarra eléctrica en escenario
    \item La exploración colectiva en ensambles grandes, a partir de lenguajes musicales contemporáneos
    \item El uso de referentes precisos sobre los cuales desarrollar mi obra.
\end{enumerate}



\clearpage

\printbibliography[title={Bibliografía}, heading=bibintoc]

\end{document}