\documentclass{article}

\usepackage[spanish]{babel}

% Required for inserting images
\usepackage{graphicx} 

\usepackage{listings}

\usepackage[utf8]{inputenc}
\usepackage{glossaries}
\usepackage{csquotes}
\usepackage{parskip}

%Imports biblatex package
\usepackage{biblatex}

\usepackage{hyperref}
\hypersetup{
    colorlinks,
    citecolor=black,
    filecolor=black,
    linkcolor=black,
    urlcolor=black
}

% import the bibliography file
\addbibresource{misticaMusicaBib.bib}

\setlength{\parindent}{0pt}

\title{Mística y música algorítmica}

\author{Aarón Montoya-Moraga}

\date{Julio 2025}

\begin{document}

\maketitle

\renewcommand*\contentsname{Tabla de contenidos}

\tableofcontents

\clearpage

\section{Introducción}

Este documento es el resultado de la investigación que realicé durante el seminario de investigación \textit{Mística y música}, dictado por el profesor y doctor Felipe Cussen en IDEA USACH, en el marco del primer semestre del Doctorado de Artes y Humanidades, entre marzo y julio 2025.

Enfrenté este curso desde mi gran ignorancia, que iba desde la diferencia entre mística y religión, la historia sufí y mis superficiales nociones sobre el movimiento \textit{new age}.

En este texto primero describo mis definiciones sobre mística, música y algoritmos, basándome en la bibliografía del curso. 

A continuación documento el trabajo y la experiencia detrás de mi participación en el concierto "Acordes Místicos" en la Iglesia de la Veracruz, junto a las colegas del curso.

En este concierto apliqué mi aproximación sensible al uso de la guitarra eléctrica como una pieza fundamental de mi práctica artística, sonora y musical.

Este es el tercero de los tres textos que escribí como cierre de los tres cursos del primer semestre del doctorado, porque me iba ser el más difícil de aterrizar.

\clearpage

\section{Definición de mística}

Para definir traición, en este seminario de investigación estudiamos una nutrida bibliografía de autores, entre las que destaco las expuestas aquí.

Åkerström explica en la introducción de su libro \textit{Betrayal and Betrayers: The Sociology of Treachery} , que la razón de escribir esta obra es debido a la importancia del tema y su abandono:

\begin{quote}
This book is thus an attempt to explore and illuminate a very basic and general social phenomenon that has been neglected.\cite[p. x]{betrayalBetrayersAkerstrom}

\end{quote}

Más adelante Åkerström postula que los eventos que etiquetamos como traiciones implican \textit{breaches of trust} \cite[p. 1]{betrayalBetrayersAkerstrom}, lo que a español podemos traducir como incumplimientos o rupturas de confianza.

Notemos que el concepto \textit{breach} también se usa en contextos computacionales, para indicar vulneraciones de seguridad, como un ataque informático, o una filtración de datos sensibles.

En su tesis doctoral, el autor Jacoby escribe:

\begin{quote}
In this thesis I analyse the concepts of trust, trustworthiness, and betrayal, and
explicate the interplay between them. \cite[p. 1]{trustBetrayalJacoby}
\end{quote}

En esta proposición quiero destacar que la traición no es un concepto que se puede definir de forma aislada, también debemos hacernos cargo de las definiciones de confianza, integridad, entre otros.

Emerge el concepto de \textit{interplay}, que en español podemos  a español como interacciones, pero creo que se pierde en la traducción la partícula \textit{play} de juego. Esta componente lúdica también se pierde en la traducción de actividades artísticas, como en \textit{play the guitar} que se convierte en tocar la guitarra.

Basándome en las lecturas y en mi entendimiento de la traición, la definiré como una coreografía donde dos entidades bailan los siguientes pasos en este orden:

\begin{enumerate}
    \item La entidad 1 decide que la entidad 2 es digna de su confianza.
    \item La entidad 1 pide depositar su confianza en la entidad 2.
    \item La entidad 2 consiente ese depósito y lo recibe.
    \item La entidad 1 y la entidad 2 se convierten en un nosotres.
    \item La posibilidad de la traición aparece.
    \item La entidad 2 rompe la confianza de la entidad 1.
    \item Aparece la traición, independiente de si la entidad 1 se entera en el momento, más tarde, o incluso nunca.
\end{enumerate}

Durante este texto trataré a las entidades como personas individuales, no como entidades mayores como estados, naciones, o religiones.

\clearpage

\section{Definición de administración}

\clearpage

\section{Definición de traición computacional}


\clearpage

\section{Administración de traición computacional}

\clearpage

\section{Conclusiones y pasos a seguir}

Espero tras este curso continuar con tres aspectos desarrollados durante el curso:

\begin{enumerate}
    \item El uso de la guitarra eléctrica en escenario
    \item La exploración colectiva en ensambles grandes, a partir de lenguajes musicales contemporáneos
    \item El uso de referentes precisos sobre los cuales desarrollar mi obra.
\end{enumerate}



\clearpage

\printbibliography[title={Bibliografía}, heading=bibintoc]

\end{document}