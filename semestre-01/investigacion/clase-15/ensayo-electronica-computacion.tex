\documentclass{article}

\usepackage[spanish]{babel}

% Required for inserting images
\usepackage{graphicx} 

\usepackage[utf8]{inputenc}
\usepackage{glossaries}
\usepackage{csquotes}
\usepackage{parskip}

%Imports biblatex package
\usepackage{biblatex}

\usepackage{hyperref}
\hypersetup{
    colorlinks,
    citecolor=black,
    filecolor=black,
    linkcolor=black,
    urlcolor=black
}

% import the bibliography file
\addbibresource{investigacion.bib}

\setlength{\parindent}{0pt}

\title{Una propuesta para la administración computacional de traiciones}
\author{Aarón Montoya-Moraga}

\date{Julio 2025}

\begin{document}

\maketitle

\renewcommand*\contentsname{Tabla de contenidos}

\tableofcontents

\clearpage


\section{Introducción}

Este ensayo es la primera versión de un manifiesto y matriz de estudios realizada durante el curso de investigación inter y transdisciplinar, dictado por las profesoras y doctoras Cynthia Shuffer y Carolina Pizarro, en IDEA USACH, en el marco del primer semestre del Doctorado de Artes y Humanidades, entre marzo y julio 2025.

En este texto enuncio el lugar en el que emplazo mi mi investigación doctoral en las áreas de arte electrónico y computacional, incluyendo definiciones y antecedentes que considero relevantes.

En este texto trato de enumerar y sistematizar las formas y enfoques con que concibo mi práctica de investigación trandisciplinaria, artística y académica.

\clearpage

\section{Mal uso de la palabra tecnología}

Se suele usar la palabra \textit{tecnología} como sinónimo de estas palabras:

\begin{enumerate}
    \item Computadores
    \item Internet
    \item Software
    \item Circuitos eléctricos
    \item Inteligencia artificial
    \item Robótica
    \item Fabricación digital
    \item Modelado por deposición fundida
    \item Realidad virtual
\end{enumerate}

Estas tecnologías son parte de mi práctica artística, pero son solamente un subconjunto de las tecnologías existentes.

Tengo la sospecha que se usa la palabra tecnología como un sinónimo de los adjetivos\textit{computacional}, \textit{virtual}, \textit{electrónico}, \textit{digital}, y en general lo que sea \textit{nuevo}.

Elevo una plegaria para que usemos la palabra tecnología en un sentido amplio si es necesario, pero no como sinónimo de las palabras expuestas.

Como la tecnología es la aplicación de conocimientos conceptuales para el logro de metas prácticas \cite{technology}, en mi taller me rodean herramientas de uso diario que también son tecnología, incluyendo:

\begin{enumerate}
    \item Lápices
    \item Escobas
    \item Destornilladores
    \item Papeles
    \item Cucharas
    \item Bicicletas
    \item Guitarras
    \item Cautines
    \item Pinzas
\end{enumerate}

Cuando digo tecnologías incluyo a todas las tecnologías, cuando digo electrónico me refiero a lo electrónico, cuando digo computacional me refiero a lo computacional.

\clearpage


\section{Conclusiones y pasos a seguir}

Durante mi investigación doctoral me enfoco en diseño y construcción de instrumentos musicales electrónicos y computacionales.

Para defender este lugar transdisciplinar, necesito crear comunidad, por eso con mucha alegría quiero compartir que en 2023 y 2024 en mi rol de profesore asistente participé de la creación de la nueva malla curricular de la Escuela de Diseño de la Universidad Diego Portales. Quiero destacar que creamos la nueva mención de diseño en interacción digital, para abordar los intereses disciplinares locales y mundiales en el diseño de interfaces electrónicas, de experiencias de usuario, de robótica y software.

Esta malla se implementó este 2025, y el próximo año cuando les estudiantes estén en su tercer semestre tendrán un nuevo curso: Pensamiento computacional, donde aprenderán los fundamentos y rudimentos de la programación, a la par de herramientas disciplinares tan importantes como lo son el dibujo, la tipografía y la geometría.

En este contexto también comparto que este año 2025 empecé a colaborar sistemáticamente en mi labor académica y profesional con el artista Matías Serrano \cite{misaa}, artista y académico chileno.

El primer semestre del año 2025 dictamos en conjunto un curso semestral en la misma Escuela de Diseño UDP: DIS8644 Taller de diseño de máquinas electrónicas. Las estudiantes aprendieron a sobre componentes básicos eléctricos como resistores y capacitores, y cómo valerse de chips de \textit{through-hole technology} \cite{tht}

En el segundo semestre de este mismo año continuaremos nuestra labor conjunta con el curso hermano DIS8645 Taller de diseño de máquinas computacionales. En este curso nos enfocaremos en la disciplina de construir placas con \textit{surface-mount technology} \cite{smt}, y con microcontroladores programados en el lenguaje C\texttt{++}.

Este julio y agosto 2025, creo que debido a esta nueva exposición en redes sociales de este trabajo, con Matías tomamos un cliente para el que estamos desarrollando un dispositivo electromecánico para percusión robótica. Gracias a las nuevas generaciones de diseñadoras y artistas que hemos ayudado a formar, gran parte de este proyecto particular está siendo producida y remunerada a generaciones más jóvenes de estudiantes, que esperamos se independicen y profesionalicen en artes electrónicas y computacionales.

Creo que es fundamental la creación de nuevos espacios curriculares donde les estudiantes se expongan a las artes electrónicas y computacionales, y así existan audiencias locales aún más robustas que sepan apreciar estas prácticas.

Esta práctica transdisciplinar que desarrollo hoy desde Chile en el año 2025 ha sido quizás anecdótica y definitivamente privilegiada, y espero que con mi labor académica y artística pueda ser la norma y con menos barreras, para quien así lo desee.

\clearpage

\printbibliography[title={Bibliografía}, heading=bibintoc]

\end{document}