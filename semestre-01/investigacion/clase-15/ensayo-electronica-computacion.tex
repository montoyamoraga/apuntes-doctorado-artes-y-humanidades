\documentclass{article}

\usepackage[spanish]{babel}

% Required for inserting images
\usepackage{graphicx} 

\usepackage[utf8]{inputenc}
\usepackage{glossaries}
\usepackage{csquotes}
\usepackage{parskip}

%Imports biblatex package
\usepackage{biblatex}

\usepackage{hyperref}
\hypersetup{
    colorlinks,
    citecolor=black,
    filecolor=black,
    linkcolor=black,
    urlcolor=black
}

% import the bibliography file
\addbibresource{investigacion.bib}

\setlength{\parindent}{0pt}

\title{Balbuceos sobre artes electrónicas y computacionales como lugares transdisciplinares}
\author{Aarón Montoya-Moraga}

\date{Julio 2025}

\begin{document}

\maketitle

\renewcommand*\contentsname{Tabla de contenidos}

\tableofcontents

\clearpage


\section{Introducción}

Este ensayo es la primera versión de un manifiesto y matriz de estudios realizada durante el curso de investigación inter y transdisciplinar, dictado por las profesoras y doctoras Cynthia Shuffer y Carolina Pizarro, en IDEA USACH, en el marco del primer semestre del Doctorado de Artes y Humanidades, entre marzo y julio 2025.

En este texto enuncio el lugar en el que emplazo mi mi investigación doctoral en las áreas de arte electrónico y computacional, incluyendo definiciones y antecedentes que considero relevantes.

En este texto trato de enumerar y sistematizar las formas y enfoques con que concibo mi práctica de investigación trandisciplinaria, artística y académica.

\clearpage

\section{Electricidad y magnetismo}

Hasta fines del siglo XIX se consideraban los fenómenos de electricidad y magnetismo como fenómenos distintos. Fue con las publicaciones de Maxwell que se unificó su estudio como fenómenos interrelacionados, demostrándose que la electricidad promueve magnetismo, que el magnetismo produce electricidad, y que la luz a su vez es un fenómeno electromagnético.

Las ecuaciones de Maxwell no salen desde la nada ni son la respuesta definitiva: incluyen trabajo previo hecho por otros físicos como Faraday, y a su vez estas ecuaciones fueron posteriormente ordenadas por Oliver Heaviside.

Las obras de Maxwell en conjunto con las de Newton son consideradas física clásica, que nos permiten abordar el estudio de fenómenos en la tierra y nuestra cotidianeidad, pero que presentan problemas e imprecisiones a escalas mayores que la terrestre, lo que se aborda en la física contemporánea.

Quiero definir como un lugar de belleza que estas ecuaciones de Maxwell nos permitan medir y predecir distintas variables gracias a ecuaciones, pero no se encargan de explicar por qué el electromagnetismo existe.

Es más, en este momento de mi vida defino la ingeniería eléctrica como una fe, ya que nunca nadie ha visto un electrón, solo sus efectos o milagros asociados.

Estas ecuaciones de Maxwell son las bases fundamentales sobre las cuales nacieron nuevas disciplinas académicas como la ingeniería eléctrica y las ciencia de la computación, que lidian con el electromagnetismo desde distintos niveles de abstracción, y que conforman la base teórica de mis mis objetos de estudio: las artes electrónicas y computacionales.

\clearpage

\section{Ingeniería eléctrica}

La ingeniería eléctrica es una ciencia aplicada, no una ciencia exacta. En los últimos 150 años esta actividad nos ha permitido ser testigos de novedades históricas que incluyen: iluminación eléctrica pública en las calles, energía eléctrica en domicilios, calefacción eléctrica, radio, micrófonos, grabación y reproducción de elementos audiovisuales, dispositivos de transmisión inalámbrica e internet. No todo ha sido un avance o un disfrute, o ha venido desde un lugar artístico o humanista, de hecho mucha de la investigación relacionada a ingeniería en el siglo XX y XXI está asociada a departamentos de "defensa", y es nuestra labor como artistas exorcizar estos orígenes violentos de los instrumentos que amamos.

En Chile a nivel eléctrico tenemos el caso particular de que gran parte de nuestra energía  viene de las denominadas energías renovables no convencionales, incluyendo el agua en represas que es capaz de mover imanes dentro de turbinas, y así producen energía eléctrica domiciliaria, lo que nos ahorra el tener el peligro de depender de reactores nucleares o de depender solamente de carbón para nuestra electricidad.

Mi formación en pregrado fue de ingeniería eléctrica en Chile entre los años 2008 y 2013. A la usanza de esa época, incluyó a grandes rasgos dos años de cursos de matemáticas y físicas, luego tres años de introducción a ingeniería eléctrica en todas sus escalas y variantes: circuitos eléctricos, sistemas digitales, señales y sistemas, sistemas de potencia, entre otros. El final de mi formación incluyó cursos más avanzados y especializados en circuitos integrados. Los primeros cursos solían tener el nombre de \textit{análisis}, y los cursos más avanzados el nombre de \textit{diseño}, de hecho el curso de último semestre requerido para egresar se llamaba \textit{diseño eléctrico}. 

Uno de mis mentores en pregrado fue el profesor y doctor Ángel Abusleme, con quien tomé cursos de electrónica avanzada y de diseño de chips, que fueron claves en delimitar mi práctica artística y académica: no diseño chips, enseño a usarlos, no creo nuevos lenguajes de programación ni optimizo sistemas de producción, los habilito para artistas y diseñadoras.

Otro mentor importante fue el profesor y doctor Rodrigo Cádiz, de formación en pregrado en ingeniería eléctrica y música, con su doctorado realizado en el área de música computacional. Con él fui ayudante de investigación en el departamento de música, en áreas de psicoacústica y diseño de instrumentos musicales digitales. En retrospectiva esta fue mi primera experiencia académica interdisciplinar, incluso transdisciplinar, ya que recuerdo esos años de formación como un lugar en el que la tensión era académica o administrativa, pero no autoral ni dentro de mí como persona practicante de la transdisciplina.

\clearpage

\section{Electrónica}

Cuando usamos la electricidad para procesar no solamente energía, sino que también información, es cuando hablamos de electrónica.

En mi práctica artística, donde uso instrumentos musicales y mi cuerpo en el escenario para controlar y secuenciar eventos audiovisuales, mi foco de investigación en electrónica ha sido por muchos años la captura de gestualidad humana en un escenario.

Por gestualidad humana me refiero a la actividad intencional que hace un cuerpo en escena, incluyendo los movimientos de sus extremidades, el sonido que produce su movimiento, la digitación de sus dedos.

Esta práctica de captura de gestualidad puede ser también expandida a sensar otros parámetros que presenta un cuerpo al habitar el mundo, como el rimto cardíaco, sudoración, respiración, actividad cerebral, entre otros.

La información que contienen estas gestualidades y parámetros estudiados son capturados por sensores, incluyendo cámaras, termómetros, micrófonos, y otros dispositivos que convierten la señal de energía original, por ejemplo mecánica, en una señal eléctrica, para procesarla con dispositivos electrónicos.

Esta práctica de sensar está siempre en el mundo y por lo tanto presenta ruido, el cual es inescapable, y mucha de la técnica desarrollada a nivel de sensores, tiene que ver con hacerse cargo del ruido inherente en las mediciones.

Estos sensores y esta información es ingresada, procesada, almacenada y emitida por la ayuda de dispositivos electrónicos, entre los que destaco:

\begin{itemize}
    \item Amplificadores
    \item Atenuadores
    \item Inversores
    \item Compresores
    \item Secuenciadores
    \item Ecualizadores
\end{itemize}

Estos dispositivos electrónicos tienen distintos alcances, historias, sabores y orígenes, lo que permita que las técnica y tecnologías estén al servicio de las artes. Hay estéticas fuertemente definidas y situadas por la materialidad del medio, como las películas grabadas en 8mm, los videos comprimidos por Youtube, los audios grabados con celulares, las fotos tomadas con Polaroids.

Esta electrónica puede ser análoga, digital, o una mezcla, pero ninguna es capaz de capturar la realidad, o de capturar sin ruido. Su desarrollo está definido por la intersección entre percepción humana (que nos lleva a subir la resolución), y la economía (que nos lleva a bajar la resolución).

\clearpage

\section{Computación}

Por computación definiremos el acto de calcular.

Dentro del área de la computación tenemos subdisciplinas como la robótica, la automatización, la  inteligencia artificial, que a su vez se valen de artefactos como algoritmos, lenguajes de programación, software y hardware.

Es de especial atención que la computación no es exclusivamente lo referido a un computador, de hecho muchos cursos de introducción a la programación son dictados en papel, para promover el pensamiento computacional abstracto.

El imaginario de los computadores de uso actual, como en el que programo esto, mientras me conecto a una nube (otro computador), que en conjunto exportan a PDF este código, son digitales, y son los de mayor uso hoy en día. Los computadores analógicos también existen, y de hecho hay esfuerzos para seguir produciéndoles para fines educacionales, como The Analog Thing \cite{analogThing}.

En mi práctica artística y académica defino términos computacionales con metáforas prestadas desde otras disciplinas: defino un algoritmo como una coreografía de danza a seguir, o una línea de código como un formulario burocrático en que pedimos que algo ocurra, pero que no asegura que lo haga.

Quiero destacar que gran parte del vocabulario que se usa en la práctica computacional proviene del inglés, donde me atrevo a decir que estas palabras fueron escogidas, adaptadas o inventadas para transmitir un mensaje de cercanía y cotidiano, en total contraposición a los sentimientos de frialdad o tecnicismos que frecuentemente percibo en español.

\begin{itemize}
    \item Software: lo suave, lo que puede cambiar frecuentemente.
    \item Hardware: lo fijo, lo que es difícil de cambiar.
    \item Chip: pedacito, en alusión al proceso de fabricación, donde se corta un pedacito de una oblea de silicio.
    \item Script: guión, secuencia de instrucciones para distintos agentes, como en el teatro.
    \item Mouse: literalmente ratón, por su forma y su cola / cable.
    \item Bug: bichito, en honor a la polilla que se coló dentro del computador Harvard Mark II y que provocó un error.
\end{itemize}

Como expongo en este apartado, la computación es una palabra muy cargada, que defino como un lugar trandisciplinar, al incorporar en su concepción múltiples orígenes y destinos.

\clearpage

\section{Mal uso de la palabra tecnología}

Se suele usar la palabra \textit{tecnología} como sinónimo de estas palabras:

\begin{enumerate}
    \item Computadores
    \item Internet
    \item Software
    \item Circuitos eléctricos
    \item Inteligencia artificial
    \item Robótica
    \item Fabricación digital
    \item Modelado por deposición fundida
    \item Realidad virtual
\end{enumerate}

Estas tecnologías son parte de mi práctica artística, pero son solamente un subconjunto de las tecnologías existentes.

Tengo la sospecha que se usa la palabra tecnología como un sinónimo de los adjetivos\textit{computacional}, \textit{virtual}, \textit{electrónico}, \textit{digital}, y en general lo que sea \textit{nuevo}.

Elevo una plegaria para que usemos la palabra tecnología en un sentido amplio si es necesario, pero no como sinónimo de las palabras expuestas.

Como la tecnología es la aplicación de conocimientos conceptuales para el logro de metas prácticas \cite{technology}, en mi taller me rodean herramientas de uso diario que también son tecnología, incluyendo:

\begin{enumerate}
    \item Lápices
    \item Escobas
    \item Destornilladores
    \item Papeles
    \item Cucharas
    \item Bicicletas
    \item Guitarras
    \item Cautines
    \item Pinzas
\end{enumerate}

Cuando digo tecnologías incluyo a todas las tecnologías, cuando digo electrónico me refiero a lo electrónico, cuando digo computacional me refiero a lo computacional.

\clearpage

\section{Artes electrónicas y computacionales}

Fue Cádiz quien me recomendó para unirme al equipo de trabajo de la doctora y profesora María José Contreras \cite{mariaJoseContreras}, con quien colaboré en obras difusas y transdisciplanares entre performance, teatro, entre activismo y academia, donde mis roles a veces eran de software y hardware, de investigación y desarrollo.

En su artículo "La práctica como investigación: nuevas metodologías para la academia latinoamericana" \cite{practicaComoInvestigacion}, Contreras propone lo siguiente:

\begin{displayquote}
    Esta herencia precolombina es tal vez una de las razones por la que en varios países de América Latina las escuelas de formación de artistas se instalaron en las universidades a diferencia de lo que sucede por ejemplo en Europa donde los artistas tradicionalmente se han formado en academias extrauniversitarias.
\end{displayquote}



Es por esta razón que 

Las palabras que usamos para definir nuestras áreas están en estados difusos y en disputa.

Definamos el arte electrónico como el lugar geométrico que cumple con ser arte creado mediante medios electrónicos.

Esta definición no me convence porque 

Celebro a colectivas y empresas pioneras en el desarrollo de productos electrónicos para las artes computacionales, como lo son Critter and Guitari, monome, y AxiDraw.

\clearpage

\section{Instrumentos para arte electrónico}

Como ejemplos de lugares dedicados a la creación de instrumentos para arte electrónico, quiero destacar los ejemplos de Korg Berlin \cite{korgBerlin} y de Ciat-Lonbarde \cite{ciatLonbarde}.

\clearpage

\section{Academia y disciplina de arte electrónico}

¿Cómo entiendo la tensión entre disciplinas académicas y saberes arraigados?

\clearpage

\section{Investigación situada}

La práctica que ejerzo está fuertemente situada desde Latinoamérica en el siglo XXI.

En este año 2025 empecé a colaborar sistemáticamente en mi labor académica y profesional con el artista Matías Serrano \cite{misaa}, artista y académico chileno.

El primer semestre del año 2025 dictamos en conjunto un curso semestral en la Escuela de Diseño de la Universidad Diego Portales: DIS8644 Taller de diseño de máquinas electrónicas. Las estudiantes aprendieron a sobre componentes básicos eléctricos como resistores y capacitores, y cómo valerse de chips de \textit{through-hole technology} \cite{tht}

En el segundo semestre de este mismo año continuaremos nuestra labor conjunta con el curso hermano DIS8645 Taller de diseño de máquinas computacionales. En este curso nos enfocaremos en la disciplina de construir placas con \textit{surface-mount technology} \cite{smt}, y con microcontroladores programados en el lenguaje C\texttt{++}.

Los artefactos que estudio existen desde el siglo XX.

El ecosistema que habito tiene que ver con electricidad doméstica, con baterías recargables, con desarrollo de chips de silicio, con la base de las leyes de Maxwell y de la computación descrita por Ada Lovelace y Alan Turing.

¿Cómo comprendo la investigación a través de la práctica?

La investigación es una caminata a través de caminos que se documentan como mejor podemos, y quedan como bitácora de lo recorrido.

\clearpage

\section{Colaboración y fuente abierta}

Tengo la postura de que nada que valga la pena es rápido de hacer, ni se hace en solitario.

Creo que los límites de los departamentos de las universidades deben ser difusos y colaborativos, y que en contextos como bandas musicales, podemos efectivamente tener roles de instrumentos definidos de base, pero que siempre pueden ser ampliados, reemplazados o rotados.

¿Cómo comprendo el movimiento disciplinar que tiene que ver con la colaboración y/o la difuminación de las fronteras disciplinares?

Si hasta Inti-Illimani se separa en facciones, qué queda para el resto.

\clearpage

\section{Conclusiones y pasos a seguir}

Durante mi investigación doctoral en diseño y construcción de instrumentos musicales electrónicos y computacionales.

\clearpage

\printbibliography[title={Bibliografía}, heading=bibintoc]

\end{document}