\documentclass{article}

\usepackage[spanish]{babel}

% Required for inserting images
\usepackage{graphicx} 

\usepackage{listings}

\usepackage[utf8]{inputenc}
\usepackage{glossaries}
\usepackage{csquotes}
\usepackage{parskip}

%Imports biblatex package
\usepackage{biblatex}

\usepackage{hyperref}
\hypersetup{
    colorlinks,
    citecolor=black,
    filecolor=black,
    linkcolor=black,
    urlcolor=black
}

% import the bibliography file
\addbibresource{misticaMusicaBib.bib}

\setlength{\parindent}{0pt}

\title{Mística y música algorítmica}

\author{Aarón Montoya-Moraga}

\date{Julio 2025}

\begin{document}

\maketitle

\renewcommand*\contentsname{Tabla de contenidos}

\tableofcontents

\clearpage

\section{Introducción}

Este documento es el resultado de la investigación que realicé durante el seminario de investigación \textit{Mística y música}, dictado por el profesor y doctor Felipe Cussen en IDEA USACH, en el marco del primer semestre del Doctorado de Artes y Humanidades, entre marzo y julio 2025.

Enfrenté este curso desde mi gran ignorancia, que iba desde la diferencia entre mística y religión, la historia sufí y mis superficiales nociones sobre el movimiento \textit{new age}.

En este texto primero describo mis definiciones sobre mística, música y algoritmos, basándome en la bibliografía del curso. 

A continuación documento el trabajo de preparación y durante mi participación en el concierto \textit{Acordes Místicos} en la Iglesia de la Veracruz, junto a las colegas del curso.

En este concierto apliqué mi aproximación sensible al uso de la guitarra eléctrica como una pieza fundamental de mi práctica artística, sonora y musical.

Este es el tercero de los tres textos que escribí como cierre de los tres cursos del primer semestre del doctorado, porque me iba ser el más difícil de aterrizar.

Durante la escritura de este texto escuché las obras Ecstatic Computation de Caterina Barbieri (2019), Music for 18 Musicians de Steve Reich (1978) y el disco Scriabin Sonatas Nos 1-10, interpretadas por el pianista Anatol Ugorski y publicadas por el sello Deutsche Gramophon(2023).

\clearpage

\section{Definición de mística}

Tras las lecturas de este curso, me atrevo a definir mística como una experiencia personal, de nuestro mundo interior, con una dimensión sagrada y fuera del cuerpo.

Los discos \textit{Ecstatic Computation} de Caterina Barbieri y \textit{Music for 18 musicians} de Steve Reich me permiten entrar en estados de trance, de meditación, y de abandono del tiempo. Destaco la ausencia de voz y palabras en estas obras, que me permiten olvidarme de ecos textuales o culturales.

A pesar de ser obras interpretadas con instrumentos musicales de mi interés, no me detengo a imaginar cómo son producidos los sonidos, ni qué marca o tipo de instrumento es el usado, solamente me enfoco en la experiencia de escucha.

Como mis obras favoritas, en momentos sensibles de escucha, en mi cuerpo siento lo fugaz y mágico de la experiencia de escucha, al ser una anécdota estadística el hecho de estar con vida, el misterio de la fuerza de gravedad y de cómo funciona el cerebro que me permite sentir el fenómeno acústico.

La escucha es también un momento de comunión con los aparatos que me permiten escuchar, como un parlante, audífonos, vinilos, agujas, tornamesas, radios, mis vecinos. Esta relación es de maravillamiento y de irrepetibilidad, ya que cada uno de estos aparatos se desintegra y sucumbe ante la entropía cada vez que emite un sonido, destruyendo también así mis sensores de escucha ante eventos sonoros de alta energía.

Este sentimiento de entrega total y de maravillamiento no me atrevo a describirlo como una experiencia mística, pero quizás es lo más cercano que siento a lo definido como mística, además de cuando estoy en trance andando en bicicleta por la ciudad, poniendo atención a cómo mantenerme en equilibrio y navegar lo desconocido.

\clearpage

\section{Definición de música}

Llevo una década de trabajo en sonido desmarcándome de la música, prefiriendo abordar la etiqueta de arte sonoro.

Como arte sonoro entiendo el arte que habita en el sonido, esto es, oscilaciones de presión que al ser percibidas por el oído y el cuerpo en general, luego son procesadas por nuestro sistema auditivo.

La música viene a ser un subconjunto de las infinitas posibilidades del arte sonoro, aplicándole una capa más definida y restrictiva de cultura.

Estas definiciones pueden hacerse metáfora, siendo el arte sonoro una entrada a un filtro llamado cultura, y a cuya salida le llamamos música.

Durante este curso investigué nuevamente sobre música, esto es, cultura. Las primeras lecturas y ejemplos vistos en clases sobre música sufí fueron reveladores, ya que me permitieron entender las influencias detrás artistas como La Monte Young (no confundir con La Meme Young).

Estudiar las influencias cercanas y lejanas de artistas que admiro ha sido central en la difuminación de mi educación sonora. Tras disfrutar a The Strokes y Sonic Young, al investigar sus referentes conocí a The Velvet Underground y Glenn Branca, y a su vez la música \textit{drone}, que fuertemente influenció mis presentaciones en vivo con guitarra eléctrica y electrónica.

Cada artista y su obra componen un organismo complejo y en constante cambio, se resisten a definiciones, y nos desafían en este movimiento a tratar de encasillarlos en estilos particulares o movimientos.

Es extremedamente difícil ponerse de acuerdo entre personas, aún más lo es explicar el sonido usando texto, y quizás imposible por definición, transmitir lo vivido durante una experiencia mística.

Esta imposibilidad de transmitir lo místico me lleva a detenerme en mi falta de interés por las partituras donde se anotan las notas musicales y sus duraciones, ya que las veo como un pálido reflejo de la música, ni siquiera como un algoritmo, ya que no se hacen cargo de lo imperfecta que es la música, manteniéndose mi interés firme en lo intransmitible que tiene una presentación de arte sonoro, y por intransmitible, incluso podemos hablar de su mística.

\clearpage

\section{Definición de algorítmica}

La algorítmica la definiremos para efectos de este texto como un adjetivo con límites difusos que nos remite a pasos, secuencias, repeticiones, iteraciones y variables.

Un algoritmo entonces es una secuencia de instrucciones, con un inicio, quizás sin un final. Estas instrucciones pueden repetirse, cambiar su flujo según variables, incluso incluir aleatoreidad u otros procesos de manipulación.

La primera vez que me enfrenté al concepto de algoritmo fue en ciencias de la computación, pero esta práctica existe en artes de mi interés, como la performance, las artes visuales y las artes sonoras.

Como artista visual y performer, Yoko Ono en su libro Grapefruit escribe algoritmos a seguir, que no necesitan de computadores ni tampoco ser realizados para convertirse en obra, como este:

\begin{quote}
  Find a stone that is your size or weight.
  
  Crack it until it becomes fine powder.
  
  Dispose of it in the river. (a)

  Send small amounts to your friends. (b)

  Do not tell anybody what you did.

  Do not explain about the powder to the friends to whom you send.\cite{grapefruitYokoOno}
\end{quote}

Otro artista que admiro por su capacidad algorítmica, también sin computadores, es Sol Lewitt, con su serie de obras Wall Drawings:

\begin{quote}
    A wall divided vertically into six equal parts, with two of the four kinds of line directions superimposed in each part.\cite{wallDrawingsSolLewitt}
\end{quote}

Durante este curso revisité a Steve Reich, quien es catalogado como minimalista, pero él se define como un compositor interesado en lo procedural, que puede ser interpretado como lo algorítmico.

A primera lectura su obra es simple de entender en sus proceder y por lo tanto predecible, pero en su escucha emergen sonoridades que desafían las primeras impresiones, y debido a su repetivivdad y la variabilidad gradual que presentan, estas piezas son capaces de sumergirte en trances donde se disuelve el tiempo.

Otra área que revisité por su enlace con la computación fue el serialismo, que pretendo abordar durante el resto de mis estudios doctorales: creo necesario el usar microcontroladores para la composición y muestra de obras seriales.

\clearpage

\section{Preparando el concierto \textit{Acordes místicos}}

Tras la definición conceptual del concierto realizada en clases, hicimos dos visitas a terreno a la iglesia para reconocer el terreno, hacer preguntas técnicas sobre enchufes y reglas de uso del espacio, y logística de montaje y desmontaje.

Decidimos realizar un ensayo exploratorio la semana anterior, para probar sonidos, tallerear nuestras composiciones, hacer un catastro de los equipos disponibles y ponernos de acuerdo en cómo realizar las transiciones entre las distintas composiciones, y cómo cada integrante del ensamble podía aportar (o no) a cada obra.

Este ensayo fue realizado en mi hogar, y dado el alto avance de las composiciones de mis colegas, decidí dedicarme ese día a la mezcla de los instrumentos en un \textit{mixer}, y aplicar mi experiencia en sonido en vivo al servicio del resto.

Mi otra gran labor en el ensayo previo fue acompañar a mi colega Tarix Sepúlveda en el aprendizaje de su recién adquirido \textit{Korg Volca Sample 2}, en el cual cargamos sonidos desde la web \textit{freesound.org} distintos sonidos para la ocasión. Para complementar el \textit{kit} de percusión electrónica de Tárix, decidimos que ella usaría también un micrófono de contacto con efectos para acompañar las obras, en lo que sería su primer concierto como música.

Antes y después este ensayo me dediqué a componer una pequeña pieza para el cierre del concierto, con muchos caminos que recorrí que luego fueron descartados.

Debido a la decisión de nombrar a nuestro concierto como \textit{Acordes místicos}, decidí enfocar mi investigación en la historia detrás del acorde místico de Alexander Scribain.

Partí pensando en distintas estrategias armónicas para lograr construir en vivo este acorde místico a partir de una nota \textit{drone}, lo que me llevó a explorar distintas estrategias para lograr una nota \textit{drone} a partir de distintas fuentes sonoras.

Quería poder partir desde alguna nota en particular, como Do o La, para poder darle esta nota al resto de los integrantes del ensamble para poder sumarse si así lo deseaban durante mi pieza.

Partí probando con mi voz como instrumento, pero para esta ocasión lo deseché por mi dificultad en afinarme a una nota en particular. El siguiente experimento consistió en usar un \textit{EBow} \cite{EBow} para lograr la nota \textit{drone}, pero este dispositivo me ocupaba una mano, y prefería tener las manos libres para manipular otros parámetros y el \textit{mixer} durante el concierto.

Decidí probar suerte con el \textit{monome norns}, pero no iba a tener tiempo suficiente de iterar el software antes del concierto, y prefería tener un enfoque donde mis gestos y mi cuerpo fueran más visibles y más presentes al momento de la performance. Queda pendiente durante el doctorado crear scripts para este sistema de hacer sonido, que esté inspirado en lo aprendido durante el curso, como serialismo.

Durante el resto de la semana estuve programando interfaces táctilas para iPad con la app \textit{Touch OSC} \cite{hexlerTouchOSC}, para controlar el pedal de efectos \textit{Digitech Whammy} \cite{digitechWhammy}, pero no llegué a resultados que me convencieran del todo, por lo que archivé esa estrategia. La interfaz creada para controlar el Whammy me convence, pero no lo suficientemente para haberla estrenada en este concierto ni con estos fines, pero sí quiero publicarla en internet para que más artistas la usen cuando esté más depurada.

Finalmente decidí armar un conjunto de pedales de efectos que me permitieran manipular el sonido de una guitarra eléctrica, y poniendo muchísimas perillas a mi disposición para filtrar el sonido, hacer \textit{loops} y otras estrategias para formar el acorde místico de forma gradual.

Esto fue posible gracias al estudio de los pedales \textit{Chase Bliss}, que tienen una doble interfaz: como pedal con perillas y botones típicos, pero también con la capacidad de recibir señales desde pedales de expresión y vía MIDI para todos sus parámetros, expandiendo enormemente la capacidad expresiva de estos pedales.

\clearpage

\section{Obra presentada en \textit{Acordes místicos}}

Para este concierto compuse esta pieza algorítmica:

\begin{enumerate}
    \item Si aún no empiezas a tocar esta pieza, elige una primera nota cualquiera. Si ya habías empezado a tocar esta pieza, salta al siguiente paso.
    \item Toca esa nota \textit{muteando} la transientes, para así resaltando la parte estable de la envolvente.
    \item Repite el paso anterior cuantas veces quieras.
    \item Calcula la siguiente nota dentro de las seis notas del acorde místico propuesto por Alexander Scriabin. Si aún no completas las seis notas, vuelve al primer paso. Si ya pasaste por las seis notas, la pieza se acaba.
\end{enumerate}

Para efectos del concierto, decidí partir la pieza con la nota Do, con lo que el acorde místico propuesto por Scriabin resulta:

\begin{enumerate}
    \item Do: nota base
    \item Fa \#: 6 semitonos sobre la nota base
    \item Sib: 10 semitonos sobre la nota base
    \item Mi: 16 semitonos sobre la nota base
    \item La: 21 semitonos sobre la nota base
    \item Re: 26 semitonos sobre la nota base
\end{enumerate}

El ataque de cada nota lo apagué usando un pedal de volumen, y cada uno de estas notas las fui grabando con el pedal \textit{Chase Bliss Blooper}, que actúa como un \textit{looper} de 8 capas, donde cada nota tocada fue grabada en una capa distinta, que pude mantener indefinidamente sonando de forma superpuesta entre sí.

Tras completar los pasos de la pieza, en vivo decidí encontrar distintas resonancias en el espacio de la iglesia, modificando el material sonoro con un último pedal en la cadena, esta vez de ecualización, permitiendome aislar distintas bandas del sonido y gradualmente ir restándolas y apagándolas.

Esta obra fue tocada con una guitarra eléctrica Fender Telecaster de color negro para combinar con el bajo eléctrico de mi colega Martín Gubbins y tiene una significancia especial: no solamente es la primera vez que me presento en vivo con esta guitarra, sino que también es la primera vez en dos años que uso una guitarra eléctrica en vivo, tras haber pausado su uso para dedicarme exclusivamente a presentarme con sintetizadores modulares \textit{Ciat-Lonbarde}.

\clearpage

\section{Conclusiones y pasos a seguir}

Espero tras este curso continuar y profundizar en tres aspectos desarrollados durante el transcurso del semestre:

\begin{enumerate}
    \item El (retorno de mi) uso de la guitarra eléctrica en escenario
    \item La exploración colectiva en ensambles, usando instrumentos electrónicos y lenguajes musicales contemporáneos
    \item La investigación de referentes conceptuales precisos sobre los cuales desarrollar mi obra.
\end{enumerate}

Este curso me hizo encontrar un vocabulario más amplio parar explicar mi fascinación con fenómenos que experiencio en la vida cotidiana como las fuerzas gravitacionales, además de mi proceso interior de vértigo al realizar conciertos de música electrónica experimental con una fuerte componente de improvisación y azar, donde estoy totalmente presente en mi cuerpo y sensores y pretendo invitar a las audiencias a una experiencia sensorial comunitaria de trance.

Espero continuar tocando en ensambles similares e incorporar la componente de salidas a terreno y muestras de procesos regulares en el resto de mi investigación doctoral, para complementar la investigación académica y el desarrollo de producots electrónicos y computacionales.

También espero poder publicar software que permita el control y generación de instrucciones musicales inspiradas en los referentes místicos estudiados en clases, para así expandir el repertorio, vocabulario y herramientas disponibles para artistas electrónicas.

\clearpage

\printbibliography[title={Bibliografía}, heading=bibintoc]

\end{document}