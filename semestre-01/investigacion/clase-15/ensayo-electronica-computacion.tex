\documentclass{article}

\usepackage[spanish]{babel}

% Required for inserting images
\usepackage{graphicx} 

\usepackage[utf8]{inputenc}
\usepackage{glossaries}
\usepackage{csquotes}
\usepackage{parskip}

%Imports biblatex package
\usepackage{biblatex}

\usepackage{hyperref}
\hypersetup{
    colorlinks,
    citecolor=black,
    filecolor=black,
    linkcolor=black,
    urlcolor=black
}

% import the bibliography file
\addbibresource{investigacion.bib}

\setlength{\parindent}{0pt}

\title{Balbuceos sobre arte electrónico y computacional como un lugar transdisciplinar}
\author{Aarón Montoya-Moraga}

\date{Julio 2025}

\begin{document}

\maketitle

\renewcommand*\contentsname{Tabla de contenidos}

\tableofcontents

\section{Introducción}

Este ensayo es la primera versión de un manifiesto y matriz de investigación realizada durante el curso de investigación inter y transdisciplinar, dictado por las profesoras y doctoras Cynthia Shuffer y Carolina Pizarro, en IDEA USACH, en el marco del primer semestre del Doctorado de Artes y Humanidades.

En este texto enuncio el lugar en el que emplazo mi mi investigación doctoral en las áreas de arte electrónico y computacional.

\section{Electricidad y magnetismo}

Hasta fines del siglo XIX se consideraban los fenómenos de electricidad y magnetismo como fenómenos distintos. Fue con las publicaciones de Maxwell que se unificó su estudio como fenómenos interrelacionados, demostrándose que la electricidad promueve magnetismo, que el magnetismo produce electricidad, y que la luz a su vez es un fenómeno electromagnético.

Las ecuaciones de Maxwell no salen desde la nada ni son la respuesta última: incluyen trabajo previo hecho por otros físicos como Faraday, y a su vez estas ecuaciones fueron posteriormente ordenadas por Heaviside.

Las ecuaciones de Maxwell, junto con las ecuaciones de Newton, son consideradas física clásica, que nos permiten abordar el estudio de fenómenos en la tierra y nuestra cotidianidad, pero que presentan problemas a escalas mayores y que son el objeto de estudio de la física contemporánea.

Quiero definir como un lugar de belleza que estas ecuaciones de Maxwell nos permitan medir y predecir distintas variables gracias a ecuaciones diferenciales, pero no se encargan de explicar el por qué el electromagnetismo existe.

Es más, en este momento de mi vida defino la ingeniería eléctrica como una fé, ya que nunca nadie ha visto un electrón, solo sus efectos (milagros).

Estas ecuaciones de Maxwell son las bases fundamentales sobre las cuales nacieron nuevas disciplinas académicas como la ingeniería eléctrica y las ciencia de la computación, que lidian con el electromagnetismo desde distintos niveles de abstracción, y que conforman mis objetos de estudio.

\section{Ingeniería eléctrica}

La ingeniería eléctrica se encarga de aplicar de forma aproximada, no exacta, la electricidad para distintos fines.

En los últimos 150 años esta actividad nos ha permitido desde tener una explosión de novedades, entre las que destaco y no doy por sentado: iluminación pública en las calles, energía eléctrica en domicilios, calefacción, grabación y reproducción de elementos audiovisuales.

En Chile además tenemos el caso particular de que gran parte de nuestra energía eléctrica viene de las denominadas energías renovables no convencionales, incluyendo el agua en represas que es capaz de mover imanes dentro de turbinas, y así producen energía eléctrica.

\section{Electrónica}

Cuando usamos la electricidad para procesar no solamente energía, sino que también información, es cuando hablamos de electrónica.



\section{Arte electrónico}

Definamos el arte electrónico como el lugar geométrico creado por el producto cruz entre los dos vectores ortogonales arte y electrónico.

\section{Instrumentos para arte electrónico}

\section{Academia y disciplina de arte electrónico}

¿Cómo entiendo la tensión entre disciplinas académicas y saberes arraigados?

\section{Investigación situada}

La práctica que ejerzo está fuertemente situada desde Latinoamérica en el siglo XXI.

Los artefactos que estudio existen desde el siglo XX.

El ecosistema que habito tiene que ver con electricidad doméstica, con baterías recargables, con desarrollo de chips de silicio, con la base de las leyes de Maxwell y de la computación descrita por Ada Lovelace y Alan Turing.

¿Cómo comprendo la investigación a través de la práctica?

La investigación es una caminata a través de caminos que se documentan como mejor podemos, y quedan como bitácora de lo recorrido.



\section{Colaboración y fuente abierta}

¿Cómo comprendo el movimiento disciplinar que tiene que ver con la colaboración y/o la difuminación de las fronteras disciplinares?

Si hasta Inti-Illimani se separa en facciones, qué queda para el resto.

\section{Conclusiones y pasos a seguir}

Ejemplos de citas de libros.

Citar libro \cite{korgBerlin}


\printbibliography[title={Bibliografía}, heading=bibintoc]

\end{document}